% Options for packages loaded elsewhere
\PassOptionsToPackage{unicode}{hyperref}
\PassOptionsToPackage{hyphens}{url}
\PassOptionsToPackage{dvipsnames,svgnames,x11names}{xcolor}
%
\documentclass[
  letterpaper,
  DIV=11,
  numbers=noendperiod]{scrartcl}

\usepackage{amsmath,amssymb}
\usepackage{iftex}
\ifPDFTeX
  \usepackage[T1]{fontenc}
  \usepackage[utf8]{inputenc}
  \usepackage{textcomp} % provide euro and other symbols
\else % if luatex or xetex
  \usepackage{unicode-math}
  \defaultfontfeatures{Scale=MatchLowercase}
  \defaultfontfeatures[\rmfamily]{Ligatures=TeX,Scale=1}
\fi
\usepackage{lmodern}
\ifPDFTeX\else  
    % xetex/luatex font selection
\fi
% Use upquote if available, for straight quotes in verbatim environments
\IfFileExists{upquote.sty}{\usepackage{upquote}}{}
\IfFileExists{microtype.sty}{% use microtype if available
  \usepackage[]{microtype}
  \UseMicrotypeSet[protrusion]{basicmath} % disable protrusion for tt fonts
}{}
\makeatletter
\@ifundefined{KOMAClassName}{% if non-KOMA class
  \IfFileExists{parskip.sty}{%
    \usepackage{parskip}
  }{% else
    \setlength{\parindent}{0pt}
    \setlength{\parskip}{6pt plus 2pt minus 1pt}}
}{% if KOMA class
  \KOMAoptions{parskip=half}}
\makeatother
\usepackage{xcolor}
\setlength{\emergencystretch}{3em} % prevent overfull lines
\setcounter{secnumdepth}{5}
% Make \paragraph and \subparagraph free-standing
\makeatletter
\ifx\paragraph\undefined\else
  \let\oldparagraph\paragraph
  \renewcommand{\paragraph}{
    \@ifstar
      \xxxParagraphStar
      \xxxParagraphNoStar
  }
  \newcommand{\xxxParagraphStar}[1]{\oldparagraph*{#1}\mbox{}}
  \newcommand{\xxxParagraphNoStar}[1]{\oldparagraph{#1}\mbox{}}
\fi
\ifx\subparagraph\undefined\else
  \let\oldsubparagraph\subparagraph
  \renewcommand{\subparagraph}{
    \@ifstar
      \xxxSubParagraphStar
      \xxxSubParagraphNoStar
  }
  \newcommand{\xxxSubParagraphStar}[1]{\oldsubparagraph*{#1}\mbox{}}
  \newcommand{\xxxSubParagraphNoStar}[1]{\oldsubparagraph{#1}\mbox{}}
\fi
\makeatother


\providecommand{\tightlist}{%
  \setlength{\itemsep}{0pt}\setlength{\parskip}{0pt}}\usepackage{longtable,booktabs,array}
\usepackage{calc} % for calculating minipage widths
% Correct order of tables after \paragraph or \subparagraph
\usepackage{etoolbox}
\makeatletter
\patchcmd\longtable{\par}{\if@noskipsec\mbox{}\fi\par}{}{}
\makeatother
% Allow footnotes in longtable head/foot
\IfFileExists{footnotehyper.sty}{\usepackage{footnotehyper}}{\usepackage{footnote}}
\makesavenoteenv{longtable}
\usepackage{graphicx}
\makeatletter
\newsavebox\pandoc@box
\newcommand*\pandocbounded[1]{% scales image to fit in text height/width
  \sbox\pandoc@box{#1}%
  \Gscale@div\@tempa{\textheight}{\dimexpr\ht\pandoc@box+\dp\pandoc@box\relax}%
  \Gscale@div\@tempb{\linewidth}{\wd\pandoc@box}%
  \ifdim\@tempb\p@<\@tempa\p@\let\@tempa\@tempb\fi% select the smaller of both
  \ifdim\@tempa\p@<\p@\scalebox{\@tempa}{\usebox\pandoc@box}%
  \else\usebox{\pandoc@box}%
  \fi%
}
% Set default figure placement to htbp
\def\fps@figure{htbp}
\makeatother

\usepackage{booktabs}
\usepackage{longtable}
\usepackage{array}
\usepackage{multirow}
\usepackage{wrapfig}
\usepackage{float}
\usepackage{colortbl}
\usepackage{pdflscape}
\usepackage{tabu}
\usepackage{threeparttable}
\usepackage{threeparttablex}
\usepackage[normalem]{ulem}
\usepackage{makecell}
\usepackage{xcolor}
\KOMAoption{captions}{tableheading}
\makeatletter
\@ifpackageloaded{tcolorbox}{}{\usepackage[skins,breakable]{tcolorbox}}
\@ifpackageloaded{fontawesome5}{}{\usepackage{fontawesome5}}
\definecolor{quarto-callout-color}{HTML}{909090}
\definecolor{quarto-callout-note-color}{HTML}{0758E5}
\definecolor{quarto-callout-important-color}{HTML}{CC1914}
\definecolor{quarto-callout-warning-color}{HTML}{EB9113}
\definecolor{quarto-callout-tip-color}{HTML}{00A047}
\definecolor{quarto-callout-caution-color}{HTML}{FC5300}
\definecolor{quarto-callout-color-frame}{HTML}{acacac}
\definecolor{quarto-callout-note-color-frame}{HTML}{4582ec}
\definecolor{quarto-callout-important-color-frame}{HTML}{d9534f}
\definecolor{quarto-callout-warning-color-frame}{HTML}{f0ad4e}
\definecolor{quarto-callout-tip-color-frame}{HTML}{02b875}
\definecolor{quarto-callout-caution-color-frame}{HTML}{fd7e14}
\makeatother
\makeatletter
\@ifpackageloaded{caption}{}{\usepackage{caption}}
\AtBeginDocument{%
\ifdefined\contentsname
  \renewcommand*\contentsname{Table of contents}
\else
  \newcommand\contentsname{Table of contents}
\fi
\ifdefined\listfigurename
  \renewcommand*\listfigurename{List of Figures}
\else
  \newcommand\listfigurename{List of Figures}
\fi
\ifdefined\listtablename
  \renewcommand*\listtablename{List of Tables}
\else
  \newcommand\listtablename{List of Tables}
\fi
\ifdefined\figurename
  \renewcommand*\figurename{Figure}
\else
  \newcommand\figurename{Figure}
\fi
\ifdefined\tablename
  \renewcommand*\tablename{Table}
\else
  \newcommand\tablename{Table}
\fi
}
\@ifpackageloaded{float}{}{\usepackage{float}}
\floatstyle{ruled}
\@ifundefined{c@chapter}{\newfloat{codelisting}{h}{lop}}{\newfloat{codelisting}{h}{lop}[chapter]}
\floatname{codelisting}{Listing}
\newcommand*\listoflistings{\listof{codelisting}{List of Listings}}
\makeatother
\makeatletter
\makeatother
\makeatletter
\@ifpackageloaded{caption}{}{\usepackage{caption}}
\@ifpackageloaded{subcaption}{}{\usepackage{subcaption}}
\makeatother

\usepackage{bookmark}

\IfFileExists{xurl.sty}{\usepackage{xurl}}{} % add URL line breaks if available
\urlstyle{same} % disable monospaced font for URLs
\hypersetup{
  pdftitle={Identify the distribution for each model},
  pdfauthor={Morgan Gray},
  colorlinks=true,
  linkcolor={blue},
  filecolor={Maroon},
  citecolor={Blue},
  urlcolor={Blue},
  pdfcreator={LaTeX via pandoc}}


\title{Identify the distribution for each model}
\author{Morgan Gray}
\date{}

\begin{document}
\maketitle

\renewcommand*\contentsname{Contents}
{
\hypersetup{linkcolor=}
\setcounter{tocdepth}{3}
\tableofcontents
}

\section{Introduction}\label{introduction}

The objective of this analysis was to determine the most suitable
probability distribution for each of six generalized linear mixed models
(GLMMs) fit to plant observation data. Two response variable types were
considered: species richness (count data) and percent cover abundance
(continuous data). For each response variable, models were fit for three
plant groups: native species, native forb species (a subset of native
species), and non-native species.

\begin{tcolorbox}[enhanced jigsaw, rightrule=.15mm, breakable, colbacktitle=quarto-callout-note-color!10!white, left=2mm, colframe=quarto-callout-note-color-frame, coltitle=black, opacitybacktitle=0.6, colback=white, toprule=.15mm, titlerule=0mm, opacityback=0, leftrule=.75mm, bottomtitle=1mm, toptitle=1mm, title=\textcolor{quarto-callout-note-color}{\faInfo}\hspace{0.5em}{Note}, arc=.35mm, bottomrule=.15mm]

The subsequent model fitting, selection, and performance assessments are
not described here.

\end{tcolorbox}

\section{Methods}\label{methods}

I conducted a systematic analysis combining visual inspection,
statistical tests, and model comparisons to identify the most suitable
probability distributions for count and continuous data. Modular
functions were developed to handle each assessment component, including
visualization generation and statistical testing. This approach provided
multiple lines of evidence for selecting the most appropriate
distribution for subsequent analyses.

The evaluation framework included:

\begin{itemize}
\item
  Visual inspection via histograms and quantile-quantile (QQ) plots
\item
  Distribution fitting tests specific to data type:

  \begin{itemize}
  \item
    Count data: goodness-of-fit tests and pairwise Vuong tests
  \item
    Continuous data: Shapiro-Wilks test
  \end{itemize}
\end{itemize}

The influence of data transformations (standardized, log, and square
root) was also explored. Log transformation was applied to positively
skewed data or data with non-negative values, while square root
transformation was used to reduce skewness in count-like or non-negative
data. A constant of 1 was added to all values prior to log
transformation for distributions requiring strictly positive values.

\subsection{Histograms}\label{histograms}

The analysis for each data type began with visual inspection of
histograms generated from the raw data. These visualizations revealed
key distributional features, including central tendency, spread,
symmetry, and potential outliers.

For count data, histogram shapes indicated potential alignment with
common distributions like Poisson (right-skewed with single peak) or
negative binomial (right-skewed with longer tail). For continuous data,
histograms were generated for raw, standardized, log-transformed, and
square-root transformed values.

\subsection{Goodness-of-fit}\label{goodness-of-fit}

For richness (count) data, discrete goodness-of-fit tests were applied
to evaluate Poisson and negative binomial distributions. P-values from
these tests were interpreted as follows:

\begin{itemize}
\tightlist
\item
  \emph{p} \textgreater{} 0.05: The data were consistent with the
  distribution (failure to reject the null hypothesis).
\item
  \emph{p} \textless{} 0.05: The data significantly deviated from the
  distribution (rejection of the null hypothesis).
\end{itemize}

For abundance (continuous) data, the Shapiro-Wilk test was used to
evaluate normality of both raw and transformed data, using the same
significance thresholds.

\subsubsection{Vuong tests}\label{vuong-tests}

For count data, Vuong tests were used to directly compare the fit of
competing models (i.e., Poisson vs.~negative binomial). The test
produces a z-statistic and p-value, where:

\begin{itemize}
\item
  Positive z-statistics (p \textless{} 0.05): First model provides
  better fit
\item
  Negative z-statistics (p \textless{} 0.05): Second model provides
  better fit
\item
  p \textgreater{} 0.05: Models are statistically indistinguishable
\end{itemize}

\subsubsection{Quantile-quantile plots}\label{quantile-quantile-plots}

QQ plots compared observed data against theoretical distributions. While
not providing discrete test statistics, these visualizations supported
statistical test results and offered insights when data failed to fit
common distributions. I evaluated plot patterns focusing on:

\begin{itemize}
\item
  Overall adherence to the diagonal reference line
\item
  Nature of deviations (random vs.~systematic)
\item
  Patterns at distribution extremes
\end{itemize}

Points falling along the diagonal reference line indicated good
agreement between observed and theoretical distributions. Deviations
from the line, particularly systematic patterns, suggested departures
from the theoretical distribution. In ecological count data, deviations
at the extremes, often due to rare species (excess zeros) or highly
abundant counts, were examined for their implications for model
selection (e.g., zero-inflated models or negative binomial models).

For count data, QQ plots were reviewed for Poisson and negative binomial
distributions. When initial tests were inconclusive (e.g., for
non-native counts), QQ plots for alternative distributions (normal,
log-normal, gamma) were examined based on histogram shapes.

In ecological count data, deviations are often observed at the extremes,
particularly due to rare species (excess zeros) or highly abundant
counts. These patterns can aid in model selection. For example,
consistent deviations at low values might suggest the need for a
zero-inflated model, while heavy tails might favor a negative binomial
over a Poisson distribution.

\subsubsection{Additional plots for continuous
data}\label{additional-plots-for-continuous-data}

For abundance (continuous) data, QQ plots were generated for normal
distributions fit to raw, standardized, square-root transformed, and
log-transformed values. In addition to QQ plots, density plots (overlaid
on histograms), cumulative distribution function (CDF) plots, and
probability-probability (PP) plots were used to visualize the comparison
between empirical and theoretical distributions.

\section{Results}\label{results}

\subsection{Richness}\label{richness}

Preview the first 10 rows of the data table for richness (rich) to see
the column names and formats.

\begin{tcolorbox}[enhanced jigsaw, rightrule=.15mm, breakable, colbacktitle=quarto-callout-tip-color!10!white, left=2mm, colframe=quarto-callout-tip-color-frame, coltitle=black, opacitybacktitle=0.6, colback=white, toprule=.15mm, titlerule=0mm, opacityback=0, leftrule=.75mm, bottomtitle=1mm, toptitle=1mm, title=\textcolor{quarto-callout-tip-color}{\faLightbulb}\hspace{0.5em}{Tip}, arc=.35mm, bottomrule=.15mm]

Scroll to the right to see more columns.

\end{tcolorbox}

\begin{longtabu} to \linewidth {>{\raggedright}X>{\raggedleft}X>{\raggedleft}X>{\raggedleft}X>{\raggedright}X>{\raggedright}X>{\raggedleft}X>{\raggedright}X>{\raggedright}X>{\raggedright}X>{\raggedright}X>{\raggedright}X>{\raggedright}X>{\raggedright}X}
\toprule
treatment & value & value\_log & value\_sqrt & plot\_name & plot\_type & year & grazer & f\_year & f\_break & f\_new & f\_one\_yr & f\_two\_yr & met\_sub\\
\midrule
Ungrazed & 8 & 2.079442 & 2.828427 & p01 & p & 2019 & Goat & y4 & b0 & n1 & o0 & t0 & rich\_non\\
Grazed & 11 & 2.397895 & 3.316625 & p01 & p & 2019 & Goat & y4 & b0 & n1 & o0 & t0 & rich\_non\\
Ungrazed & 10 & 2.302585 & 3.162278 & p02 & p & 2019 & Goat & y4 & b0 & n1 & o0 & t0 & rich\_non\\
Grazed & 11 & 2.397895 & 3.316625 & p02 & p & 2019 & Goat & y4 & b0 & n1 & o0 & t0 & rich\_non\\
Ungrazed & 8 & 2.079442 & 2.828427 & p03 & p & 2019 & Goat & y4 & b0 & n1 & o0 & t0 & rich\_non\\
\addlinespace
Grazed & 6 & 1.791760 & 2.449490 & p03 & p & 2019 & Goat & y4 & b0 & n1 & o0 & t0 & rich\_non\\
Ungrazed & 7 & 1.945910 & 2.645751 & p04 & p & 2019 & Goat & y4 & b0 & n1 & o0 & t0 & rich\_non\\
Grazed & 6 & 1.791760 & 2.449490 & p04 & p & 2019 & Goat & y4 & b0 & n1 & o0 & t0 & rich\_non\\
Ungrazed & 7 & 1.945910 & 2.645751 & p05 & p & 2019 & Goat & y4 & b0 & n1 & o0 & t0 & rich\_non\\
Grazed & 9 & 2.197225 & 3.000000 & p05 & p & 2019 & Goat & y4 & b0 & n1 & o0 & t0 & rich\_non\\
\bottomrule
\end{longtabu}

\subsubsection{Native species}\label{native-species}

\begin{tcolorbox}[enhanced jigsaw, rightrule=.15mm, breakable, colbacktitle=quarto-callout-note-color!10!white, left=2mm, colframe=quarto-callout-note-color-frame, coltitle=black, opacitybacktitle=0.6, colback=white, toprule=.15mm, titlerule=0mm, opacityback=0, leftrule=.75mm, bottomtitle=1mm, toptitle=1mm, title={Recommended distribution}, arc=.35mm, bottomrule=.15mm]

Negative binomial

\end{tcolorbox}

\paragraph{\texorpdfstring{\textbf{Histograms}}{Histograms}}\label{histograms-1}

The histograms indicated the raw data followed a Poisson distribution,
which was expected for count data.

\begin{itemize}
\item
  Raw values: Skewed left with tail
\item
  Log-transformed values: More normal but with outliers to the far left
  (negative values)
\item
  Square root-transformed values: Kindof normal with some gaps in the
  lower range
\end{itemize}

\pandocbounded{\includegraphics[keepaspectratio]{identify-distribution_files/figure-pdf/unnamed-chunk-1-1.pdf}}

\pandocbounded{\includegraphics[keepaspectratio]{identify-distribution_files/figure-pdf/unnamed-chunk-1-2.pdf}}

\pandocbounded{\includegraphics[keepaspectratio]{identify-distribution_files/figure-pdf/unnamed-chunk-1-3.pdf}}

\paragraph{\texorpdfstring{\textbf{Goodness-of-fit
tests}}{Goodness-of-fit tests}}\label{goodness-of-fit-tests}

The goodness-of-fit test results suggested a negative binomial
distribution was a good fit for the data.

\begin{itemize}
\item
  Poisson distribution: poor (p \textless{} 0.05)
\item
  Negative binomial distribution: ok (p \textgreater{} 0.05)
\end{itemize}

\begin{verbatim}

     Goodness-of-fit test for poisson distribution

                      X^2 df     P(> X^2)
Likelihood Ratio 146.0324 16 4.721291e-23
\end{verbatim}

\pandocbounded{\includegraphics[keepaspectratio]{identify-distribution_files/figure-pdf/unnamed-chunk-2-1.pdf}}

\begin{verbatim}

     Goodness-of-fit test for nbinomial distribution

                      X^2 df   P(> X^2)
Likelihood Ratio 23.99674 15 0.06514879
\end{verbatim}

\pandocbounded{\includegraphics[keepaspectratio]{identify-distribution_files/figure-pdf/unnamed-chunk-2-2.pdf}}

\paragraph{\texorpdfstring{\textbf{Vuong
tests}}{Vuong tests}}\label{vuong-tests-1}

The Vuong test results supported a negative binomial distribution for
the data.

\begin{itemize}
\item
  Negative binomial vs.~Poisson: Negative binomial better (positive z, p
  \textless{} 0.05)
\item
  Poisson vs.~zero-inflated: No difference (p \textgreater{} 0.05)
\item
  Negative binomial vs.~zero-inflated: Negative binomial better
  (positive z, p \textless{} 0.05)
\end{itemize}

\begin{verbatim}
[1] "Vuong test: Negative Binomial vs. Poisson for native richness"
Vuong Non-Nested Hypothesis Test-Statistic: 
(test-statistic is asymptotically distributed N(0,1) under the
 null that the models are indistinguishible)
-------------------------------------------------------------
              Vuong z-statistic             H_A   p-value
Raw                     4.45714 model1 > model2 4.153e-06
AIC-corrected           4.45714 model1 > model2 4.153e-06
BIC-corrected           4.45714 model1 > model2 4.153e-06
NULL


[1] "Vuong test: Poisson vs. Zero-Inflated for native richness"
Vuong Non-Nested Hypothesis Test-Statistic: 
(test-statistic is asymptotically distributed N(0,1) under the
 null that the models are indistinguishible)
-------------------------------------------------------------
              Vuong z-statistic             H_A    p-value
Raw                  -0.4419274 model2 > model1   0.329271
AIC-corrected         1.2870107 model1 > model2   0.099045
BIC-corrected         4.6220746 model1 > model2 1.8996e-06
NULL


[1] "Vuong test: Negative Binomial vs. Zero-Inflated for native richness"
Vuong Non-Nested Hypothesis Test-Statistic: 
(test-statistic is asymptotically distributed N(0,1) under the
 null that the models are indistinguishible)
-------------------------------------------------------------
              Vuong z-statistic             H_A    p-value
Raw                    4.485144 model1 > model2 3.6432e-06
AIC-corrected          4.659626 model1 > model2 1.5839e-06
BIC-corrected          4.996195 model1 > model2 2.9236e-07
NULL
\end{verbatim}

\paragraph{\texorpdfstring{\textbf{QQ
plot(s)}}{QQ plot(s)}}\label{qq-plots}

The QQ plot supported a negative binomial distribution for the data. The
points approximately followed the line, although there was some
zero-inflation (lower values above the line) and skew above the line
with upper values.

\pandocbounded{\includegraphics[keepaspectratio]{identify-distribution_files/figure-pdf/unnamed-chunk-4-1.pdf}}

\subsubsection{Native forb species}\label{native-forb-species}

\begin{tcolorbox}[enhanced jigsaw, rightrule=.15mm, breakable, colbacktitle=quarto-callout-note-color!10!white, left=2mm, colframe=quarto-callout-note-color-frame, coltitle=black, opacitybacktitle=0.6, colback=white, toprule=.15mm, titlerule=0mm, opacityback=0, leftrule=.75mm, bottomtitle=1mm, toptitle=1mm, title={Recommended distribution}, arc=.35mm, bottomrule=.15mm]

Negative binomial

\end{tcolorbox}

The results for native forb species richness were similar to those for
all native species. This was not surprising given the forb data was a
large component of the native species data.

\paragraph{\texorpdfstring{\textbf{Histograms}}{Histograms}}\label{histograms-2}

The histograms indicated the raw data followed a Poisson distribution,
which was expected for count data.

\begin{itemize}
\item
  Raw values: Skewed left with tail
\item
  Log-transformed values: More normal but with outliers to the far left
  (negative values)
\item
  Square root-transformed values: Kindof normal with some gaps in the
  lower range
\end{itemize}

\pandocbounded{\includegraphics[keepaspectratio]{identify-distribution_files/figure-pdf/unnamed-chunk-5-1.pdf}}

\pandocbounded{\includegraphics[keepaspectratio]{identify-distribution_files/figure-pdf/unnamed-chunk-5-2.pdf}}

\pandocbounded{\includegraphics[keepaspectratio]{identify-distribution_files/figure-pdf/unnamed-chunk-5-3.pdf}}

\paragraph{\texorpdfstring{\textbf{Goodness-of-fit
tests}}{Goodness-of-fit tests}}\label{goodness-of-fit-tests-1}

The goodness-of-fit test results suggested a negative binomial
distribution was a good fit for the data.

\begin{itemize}
\item
  Poisson distribution: poor (p \textless{} 0.05)
\item
  Negative binomial distribution: good (p \textgreater{} 0.05); better
  than for native species
\end{itemize}

\begin{verbatim}

     Goodness-of-fit test for poisson distribution

                      X^2 df     P(> X^2)
Likelihood Ratio 149.1477 15 3.561991e-24
\end{verbatim}

\pandocbounded{\includegraphics[keepaspectratio]{identify-distribution_files/figure-pdf/unnamed-chunk-6-1.pdf}}

\begin{verbatim}

     Goodness-of-fit test for nbinomial distribution

                      X^2 df  P(> X^2)
Likelihood Ratio 16.55131 14 0.2808772
\end{verbatim}

\pandocbounded{\includegraphics[keepaspectratio]{identify-distribution_files/figure-pdf/unnamed-chunk-6-2.pdf}}

\paragraph{\texorpdfstring{\textbf{Vuong
tests}}{Vuong tests}}\label{vuong-tests-2}

The Vuong test results supported a negative binomial distribution for
the data.

\begin{itemize}
\item
  Negative binomial vs.~Poisson: Negative binomial better (positive z, p
  \textless{} 0.05)
\item
  Poisson vs.~zero-inflated: No difference (p \textgreater{} 0.05)
\item
  Negative binomial vs.~zero-inflated: Negative binomial better
  (positive z, p \textless{} 0.05)
\end{itemize}

\begin{verbatim}
[1] "Vuong test: Negative Binomial vs. Poisson for native forb richness"
Vuong Non-Nested Hypothesis Test-Statistic: 
(test-statistic is asymptotically distributed N(0,1) under the
 null that the models are indistinguishible)
-------------------------------------------------------------
              Vuong z-statistic             H_A    p-value
Raw                    4.284394 model1 > model2 9.1619e-06
AIC-corrected          4.284394 model1 > model2 9.1619e-06
BIC-corrected          4.284394 model1 > model2 9.1619e-06
NULL


[1] "Vuong test: Poisson vs. Zero-Inflated for native forb richness"
Vuong Non-Nested Hypothesis Test-Statistic: 
(test-statistic is asymptotically distributed N(0,1) under the
 null that the models are indistinguishible)
-------------------------------------------------------------
              Vuong z-statistic             H_A  p-value
Raw                  -1.4491151 model2 > model1 0.073653
AIC-corrected        -1.0565711 model2 > model1 0.145354
BIC-corrected        -0.2993668 model2 > model1 0.382330
NULL


[1] "Vuong test: Negative Binomial vs. Zero-Inflated for native forb richness"
Vuong Non-Nested Hypothesis Test-Statistic: 
(test-statistic is asymptotically distributed N(0,1) under the
 null that the models are indistinguishible)
-------------------------------------------------------------
              Vuong z-statistic             H_A    p-value
Raw                    4.047711 model1 > model2 2.5861e-05
AIC-corrected          4.222072 model1 > model2 1.2103e-05
BIC-corrected          4.558408 model1 > model2 2.5771e-06
NULL
\end{verbatim}

\paragraph{\texorpdfstring{\textbf{QQ
plot(s)}}{QQ plot(s)}}\label{qq-plots-1}

The QQ plot supported a negative binomial distribution for the data. The
points approximately followed the line, although there was some
zero-inflation (lower values above the line).

\pandocbounded{\includegraphics[keepaspectratio]{identify-distribution_files/figure-pdf/unnamed-chunk-8-1.pdf}}

\subsubsection{Non-native species}\label{non-native-species}

\begin{tcolorbox}[enhanced jigsaw, rightrule=.15mm, breakable, colbacktitle=quarto-callout-note-color!10!white, left=2mm, colframe=quarto-callout-note-color-frame, coltitle=black, opacitybacktitle=0.6, colback=white, toprule=.15mm, titlerule=0mm, opacityback=0, leftrule=.75mm, bottomtitle=1mm, toptitle=1mm, title={Recommended distribution}, arc=.35mm, bottomrule=.15mm]

Normal (with log-transformed values)

\end{tcolorbox}

\paragraph{\texorpdfstring{\textbf{Histograms}}{Histograms}}\label{histograms-3}

The histogram for the square-root transformed data appeared the most
normal of those considered.

\begin{itemize}
\item
  Raw values: Slightly skewed to the left
\item
  Standardized: Slightly skewed to the left
\item
  Log-transformed values: Skewed to the right
\item
  Square root-transformed values: Kindof normal with some gaps in the
  lower range
\end{itemize}

\pandocbounded{\includegraphics[keepaspectratio]{identify-distribution_files/figure-pdf/unnamed-chunk-9-1.pdf}}

\pandocbounded{\includegraphics[keepaspectratio]{identify-distribution_files/figure-pdf/unnamed-chunk-9-2.pdf}}

\pandocbounded{\includegraphics[keepaspectratio]{identify-distribution_files/figure-pdf/unnamed-chunk-9-3.pdf}}

\pandocbounded{\includegraphics[keepaspectratio]{identify-distribution_files/figure-pdf/unnamed-chunk-9-4.pdf}}

\paragraph{\texorpdfstring{\textbf{Goodness-of-fit
tests}}{Goodness-of-fit tests}}\label{goodness-of-fit-tests-2}

The goodness-of-fit test results suggested a Poisson distribution was an
ok fit for the raw data.

\begin{itemize}
\item
  Poisson distribution: ok (p \textgreater{} 0.05)
\item
  Negative binomial distribution: poor (p \textless{} 0.05)
\end{itemize}

\begin{verbatim}

     Goodness-of-fit test for poisson distribution

                      X^2 df   P(> X^2)
Likelihood Ratio 21.87264 14 0.08126932
\end{verbatim}

\pandocbounded{\includegraphics[keepaspectratio]{identify-distribution_files/figure-pdf/unnamed-chunk-10-1.pdf}}

\begin{verbatim}

     Goodness-of-fit test for nbinomial distribution

                      X^2 df   P(> X^2)
Likelihood Ratio 23.24761 13 0.03879904
\end{verbatim}

\pandocbounded{\includegraphics[keepaspectratio]{identify-distribution_files/figure-pdf/unnamed-chunk-10-2.pdf}}

\paragraph{\texorpdfstring{\textbf{Vuong
tests}}{Vuong tests}}\label{vuong-tests-3}

The Vuong test results supported a Poisson distribution for the data.

\begin{itemize}
\tightlist
\item
  Negative binomial vs.~Poisson: Poisson better (negative z, p
  \textless{} 0.05)
\end{itemize}

\begin{verbatim}
[1] "Vuong test: Negative Binomial vs. Poisson for non-native richness"
Vuong Non-Nested Hypothesis Test-Statistic: 
(test-statistic is asymptotically distributed N(0,1) under the
 null that the models are indistinguishible)
-------------------------------------------------------------
              Vuong z-statistic             H_A   p-value
Raw                   -2.538273 model2 > model1 0.0055701
AIC-corrected         -2.538273 model2 > model1 0.0055701
BIC-corrected         -2.538273 model2 > model1 0.0055701
NULL
\end{verbatim}

\paragraph{\texorpdfstring{\textbf{QQ
plot(s)}}{QQ plot(s)}}\label{qq-plots-2}

The QQ plot for the Poisson distribution was not a strong fit for the
data. The points approximately followed the line, although never within
the bounds; lower values were below the line and upper values were above
line.

\pandocbounded{\includegraphics[keepaspectratio]{identify-distribution_files/figure-pdf/unnamed-chunk-12-1.pdf}}

\begin{center}\rule{0.5\linewidth}{0.5pt}\end{center}

\subsection{Abundance}\label{abundance}

Preview the first 10 rows of the data table for abundance (abun) to see
the column names and formats.

\begin{tcolorbox}[enhanced jigsaw, rightrule=.15mm, breakable, colbacktitle=quarto-callout-tip-color!10!white, left=2mm, colframe=quarto-callout-tip-color-frame, coltitle=black, opacitybacktitle=0.6, colback=white, toprule=.15mm, titlerule=0mm, opacityback=0, leftrule=.75mm, bottomtitle=1mm, toptitle=1mm, title=\textcolor{quarto-callout-tip-color}{\faLightbulb}\hspace{0.5em}{Tip}, arc=.35mm, bottomrule=.15mm]

Scroll to the right to see more columns.

\end{tcolorbox}

\begin{longtabu} to \linewidth {>{\raggedright}X>{\raggedleft}X>{\raggedleft}X>{\raggedleft}X>{\raggedleft}X>{\raggedright}X>{\raggedright}X>{\raggedleft}X>{\raggedright}X>{\raggedright}X>{\raggedright}X>{\raggedright}X>{\raggedright}X>{\raggedright}X>{\raggedright}X}
\toprule
treatment & value & value\_log & value\_std & value\_sqrt & plot\_name & plot\_type & year & grazer & f\_year & f\_break & f\_new & f\_one\_yr & f\_two\_yr & met\_sub\\
\midrule
Ungrazed & 66 & 4.189655 & 0.4768881 & 8.124038 & p01 & p & 2019 & Goat & y4 & b0 & n1 & o0 & t0 & abun\_non\\
Ungrazed & 85 & 4.442651 & 1.0742334 & 9.219544 & p02 & p & 2019 & Goat & y4 & b0 & n1 & o0 & t0 & abun\_non\\
Ungrazed & 71 & 4.262680 & 0.6340842 & 8.426150 & p03 & p & 2019 & Goat & y4 & b0 & n1 & o0 & t0 & abun\_non\\
Ungrazed & 86 & 4.454347 & 1.1056726 & 9.273618 & p04 & p & 2019 & Goat & y4 & b0 & n1 & o0 & t0 & abun\_non\\
Ungrazed & 51 & 3.931826 & 0.0052998 & 7.141428 & p05 & p & 2019 & Goat & y4 & b0 & n1 & o0 & t0 & abun\_non\\
\addlinespace
Ungrazed & 45 & 3.806662 & -0.1833356 & 6.708204 & p06 & p & 2019 & Goat & y4 & b0 & n1 & o0 & t0 & abun\_non\\
Ungrazed & 63 & 4.143135 & 0.3825704 & 7.937254 & p07 & p & 2019 & Goat & y4 & b0 & n1 & o0 & t0 & abun\_non\\
Ungrazed & 98 & 4.584968 & 1.4829433 & 9.899495 & p08 & p & 2019 & Goat & y4 & b0 & n1 & o0 & t0 & abun\_non\\
Ungrazed & 29 & 3.367296 & -0.6863632 & 5.385165 & p09 & p & 2019 & Goat & y4 & b0 & n1 & o0 & t0 & abun\_non\\
Ungrazed & 107 & 4.672829 & 1.7658963 & 10.344080 & p10 & p & 2019 & Goat & y4 & b0 & n1 & o0 & t0 & abun\_non\\
\bottomrule
\end{longtabu}

\subsubsection{Native species}\label{native-species-1}

\begin{tcolorbox}[enhanced jigsaw, rightrule=.15mm, breakable, colbacktitle=quarto-callout-note-color!10!white, left=2mm, colframe=quarto-callout-note-color-frame, coltitle=black, opacitybacktitle=0.6, colback=white, toprule=.15mm, titlerule=0mm, opacityback=0, leftrule=.75mm, bottomtitle=1mm, toptitle=1mm, title={Recommended distribution}, arc=.35mm, bottomrule=.15mm]

Normal (with log-transformed values)

\end{tcolorbox}

\paragraph{\texorpdfstring{\textbf{Histograms}}{Histograms}}\label{histograms-4}

The histograms indicated the log-transformed data most closely resembled
a normal distribution.

\begin{itemize}
\item
  Raw values: Skewed left, long tail to right
\item
  Standardized values: Skewed left between -1 to 0, long tail to right
\item
  Log-transformed values: Normal distribution with outlier to far left
  (negative)
\item
  Square root-transformed values: Main peak at 3, additional peak(s)
  between 7-10
\end{itemize}

\pandocbounded{\includegraphics[keepaspectratio]{identify-distribution_files/figure-pdf/unnamed-chunk-13-1.pdf}}

\pandocbounded{\includegraphics[keepaspectratio]{identify-distribution_files/figure-pdf/unnamed-chunk-13-2.pdf}}

\pandocbounded{\includegraphics[keepaspectratio]{identify-distribution_files/figure-pdf/unnamed-chunk-13-3.pdf}}

\pandocbounded{\includegraphics[keepaspectratio]{identify-distribution_files/figure-pdf/unnamed-chunk-13-4.pdf}}

\paragraph{Goodness-of-fit}\label{goodness-of-fit-1}

A comparison of theoretical and empirical distributions showed the best
fit for log-transformed values.

\begin{itemize}
\item
  Raw values: Different central tendency and spread for empirical and
  theoretical density plot; curvature away from the reference line in
  the QQ plot, PP plot, and CDF. The Shapiro-Wilk normality test results
  for indicated the data significantly deviated from a normal
  distribution.
\item
  Standardized values: Same as for raw values
\item
  Log-transformed values: Similar central tendency and spread for
  density curves; points relatively consistent with reference line for
  QQ plot, PP plot, and CDF. The Shapiro-Wilk normality test results for
  indicated the data significantly deviated from a normal distribution.
\item
  Square root-transformed values: Different central tendency, spread,
  and symmetry for empirical and theoretical density plot; modest
  curvature away from the reference line in the QQ plot (especially at
  low values), PP plot, and CDF. While the W statistic suggests a
  reasonable degree of normality, the very small p-value suggests that
  the deviation from normality is statistically significant.
\end{itemize}

\begin{verbatim}
[1] "Normal Distribution Fit for native abundance"

    Shapiro-Wilk normality test

data:  data$value
W = 0.81529, p-value < 2.2e-16
\end{verbatim}

\pandocbounded{\includegraphics[keepaspectratio]{identify-distribution_files/figure-pdf/unnamed-chunk-14-1.pdf}}

\begin{verbatim}

[1] "Normal Distribution Fit for Standardized native abundance"

    Shapiro-Wilk normality test

data:  data$value_std
W = 0.81529, p-value < 2.2e-16
\end{verbatim}

\pandocbounded{\includegraphics[keepaspectratio]{identify-distribution_files/figure-pdf/unnamed-chunk-14-2.pdf}}

\begin{verbatim}

[1] "Normal Distribution Fit for Log-transformed native abundance"

    Shapiro-Wilk normality test

data:  data$value_log
W = 0.68187, p-value < 2.2e-16
\end{verbatim}

\pandocbounded{\includegraphics[keepaspectratio]{identify-distribution_files/figure-pdf/unnamed-chunk-14-3.pdf}}

\begin{verbatim}

[1] "Normal Distribution Fit for Square Root-transformed native abundance"

    Shapiro-Wilk normality test

data:  data$value_sqrt
W = 0.93602, p-value = 3.981e-11
\end{verbatim}

\pandocbounded{\includegraphics[keepaspectratio]{identify-distribution_files/figure-pdf/unnamed-chunk-14-4.pdf}}

\subsubsection{Native forb species}\label{native-forb-species-1}

\begin{tcolorbox}[enhanced jigsaw, rightrule=.15mm, breakable, colbacktitle=quarto-callout-note-color!10!white, left=2mm, colframe=quarto-callout-note-color-frame, coltitle=black, opacitybacktitle=0.6, colback=white, toprule=.15mm, titlerule=0mm, opacityback=0, leftrule=.75mm, bottomtitle=1mm, toptitle=1mm, title={Recommended distribution}, arc=.35mm, bottomrule=.15mm]

Normal (with log-transformed values)

\end{tcolorbox}

The results for native forb species abundance were similar to those for
all native species. This was not surprising given the forb data was a
large component of the native species data.

\paragraph{Histograms}\label{histograms-5}

The histograms indicated the log-transformed data most closely resembled
a normal distribution.

\begin{itemize}
\item
  Raw values: Skewed left, long tail to right
\item
  Standardized values: Skewed left between -1 to 0, long tail to right
\item
  Log-transformed values: Normal distribution with outlier to far left
  (negative)
\item
  Square root-transformed values: Main peak at 3, broad plateau (no
  clear peak) of elevated counts between 4-10
\end{itemize}

\pandocbounded{\includegraphics[keepaspectratio]{identify-distribution_files/figure-pdf/unnamed-chunk-15-1.pdf}}

\pandocbounded{\includegraphics[keepaspectratio]{identify-distribution_files/figure-pdf/unnamed-chunk-15-2.pdf}}

\pandocbounded{\includegraphics[keepaspectratio]{identify-distribution_files/figure-pdf/unnamed-chunk-15-3.pdf}}

\pandocbounded{\includegraphics[keepaspectratio]{identify-distribution_files/figure-pdf/unnamed-chunk-15-4.pdf}}

\paragraph{Goodness-of-fit}\label{goodness-of-fit-2}

A comparison of theoretical and empirical distributions showed the best
fit for log-transformed values. The square root-transformed values
appeared to behave well upon visual inspection. I opted for the
log-transformation in part to be consistent with the approach used for
native species and also based on model selection results. Specifically,
I compared the performance of models fit to the log-transformed and
square-root transformed values (results not shown here) and found the
log transformation to be a better approach.

\begin{itemize}
\item
  Raw values: Different central tendency and spread for empirical and
  theoretical density plot; curvature away from the reference line in
  the QQ plot, PP plot, and CDF. The Shapiro-Wilk normality test results
  for indicated the data significantly deviated from a normal
  distribution.
\item
  Standardized values: Same as for raw values
\item
  Log-transformed values: Similar central tendency and spread for
  density curves; points relatively consistent with reference line for
  CDF; low values fall below the reference line for QQ plot and PP plot.
  The Shapiro-Wilk normality test results for indicated the data
  significantly deviated from a normal distribution.
\item
  Square root-transformed values: Different central tendency, spread,
  and symmetry for empirical and theoretical density plot; modest
  curvature away from the reference line in the QQ plot, PP plot, and
  CDF. While the W statistic suggests a reasonable degree of normality,
  the very small p-value suggests that the deviation from normality is
  statistically significant.
\end{itemize}

\begin{verbatim}
[1] "Normal Distribution Fit for native forb abundance"

    Shapiro-Wilk normality test

data:  data$value
W = 0.79162, p-value < 2.2e-16
\end{verbatim}

\pandocbounded{\includegraphics[keepaspectratio]{identify-distribution_files/figure-pdf/unnamed-chunk-16-1.pdf}}

\begin{verbatim}

[1] "Normal Distribution Fit for Standardized native forb abundance"

    Shapiro-Wilk normality test

data:  data$value_std
W = 0.79162, p-value < 2.2e-16
\end{verbatim}

\pandocbounded{\includegraphics[keepaspectratio]{identify-distribution_files/figure-pdf/unnamed-chunk-16-2.pdf}}

\begin{verbatim}

[1] "Normal Distribution Fit for Log-transformed native forb abundance"

    Shapiro-Wilk normality test

data:  data$value_log
W = 0.53144, p-value < 2.2e-16
\end{verbatim}

\pandocbounded{\includegraphics[keepaspectratio]{identify-distribution_files/figure-pdf/unnamed-chunk-16-3.pdf}}

\begin{verbatim}

[1] "Normal Distribution Fit for Square Root-transformed native forb abundance"

    Shapiro-Wilk normality test

data:  data$value_sqrt
W = 0.93002, p-value = 9.525e-12
\end{verbatim}

\pandocbounded{\includegraphics[keepaspectratio]{identify-distribution_files/figure-pdf/unnamed-chunk-16-4.pdf}}

\subsubsection{Non-native species}\label{non-native-species-1}

\begin{tcolorbox}[enhanced jigsaw, rightrule=.15mm, breakable, colbacktitle=quarto-callout-note-color!10!white, left=2mm, colframe=quarto-callout-note-color-frame, coltitle=black, opacitybacktitle=0.6, colback=white, toprule=.15mm, titlerule=0mm, opacityback=0, leftrule=.75mm, bottomtitle=1mm, toptitle=1mm, title={Recommended distribution}, arc=.35mm, bottomrule=.15mm]

Normal (with square root-transformed values)

\end{tcolorbox}

\paragraph{Histograms}\label{histograms-6}

The histograms indicated the square root-transformed data most closely
resembled a normal distribution.

\begin{itemize}
\item
  Raw values: Skewed left, tail to right
\item
  Standardized values: Skewed left between -1 to 0, tail to right
\item
  Log-transformed values: Skewed right with tail to left
\item
  Square root-transformed values: Kindof normal
\end{itemize}

\pandocbounded{\includegraphics[keepaspectratio]{identify-distribution_files/figure-pdf/unnamed-chunk-17-1.pdf}}

\pandocbounded{\includegraphics[keepaspectratio]{identify-distribution_files/figure-pdf/unnamed-chunk-17-2.pdf}}

\pandocbounded{\includegraphics[keepaspectratio]{identify-distribution_files/figure-pdf/unnamed-chunk-17-3.pdf}}

\pandocbounded{\includegraphics[keepaspectratio]{identify-distribution_files/figure-pdf/unnamed-chunk-17-4.pdf}}

\paragraph{Goodness-of-fit}\label{goodness-of-fit-3}

A comparison of theoretical and empirical distributions showed the best
fit for square root-transformed values.

\begin{itemize}
\item
  Raw values: Different central tendency and symmetry for empirical and
  theoretical density plot; curvature away from the reference line for
  upper values in the QQ plot; points relatively consistent with
  reference line for PP plot, and CDF. While the W statistic suggests a
  high degree of normality, the significant p-value indicates a
  deviation from normality.
\item
  Standardized values: Same as for raw values
\item
  Log-transformed values: Different central tendency and symmetry for
  empirical and theoretical density plot; modest curvature away from the
  reference line in the QQ plot, PP plot, and CDF. While the W statistic
  suggests a reasonable degree of normality, the very small p-value
  suggests that the deviation from normality is statistically
  significant.
\item
  Square root-transformed values: Similar central tendency and spread
  for density curves (empirical still isn't completely symmetrical);
  points relatively consistent with reference line for the QQ plot, PP
  plot, and CDF. Even though the W statistic is close to 1 (suggesting
  near-normality), the p-value (0.0059) provides evidence that the data
  significantly deviates from a normal distribution. But this was the
  best fit among the options.
\end{itemize}

\begin{verbatim}
[1] "Normal Distribution Fit for non-native abundance"

    Shapiro-Wilk normality test

data:  data$value
W = 0.95865, p-value = 2.265e-08
\end{verbatim}

\pandocbounded{\includegraphics[keepaspectratio]{identify-distribution_files/figure-pdf/unnamed-chunk-18-1.pdf}}

\begin{verbatim}

[1] "Normal Distribution Fit for Standardized non-native abundance"

    Shapiro-Wilk normality test

data:  data$value_std
W = 0.95865, p-value = 2.265e-08
\end{verbatim}

\pandocbounded{\includegraphics[keepaspectratio]{identify-distribution_files/figure-pdf/unnamed-chunk-18-2.pdf}}

\begin{verbatim}

[1] "Normal Distribution Fit for Log-transformed non-native abundance"

    Shapiro-Wilk normality test

data:  data$value_log
W = 0.92746, p-value = 5.293e-12
\end{verbatim}

\pandocbounded{\includegraphics[keepaspectratio]{identify-distribution_files/figure-pdf/unnamed-chunk-18-3.pdf}}

\begin{verbatim}

[1] "Normal Distribution Fit for Square Root-transformed non-native abundance"

    Shapiro-Wilk normality test

data:  data$value_sqrt
W = 0.98814, p-value = 0.005948
\end{verbatim}

\pandocbounded{\includegraphics[keepaspectratio]{identify-distribution_files/figure-pdf/unnamed-chunk-18-4.pdf}}

\begin{center}\rule{0.5\linewidth}{0.5pt}\end{center}

\section{Session info}\label{session-info}

\begin{verbatim}
- Session info ---------------------------------------------------------------
 setting  value
 version  R version 4.4.0 (2024-04-24)
 os       macOS 15.2
 system   aarch64, darwin20
 ui       X11
 language (EN)
 collate  en_US.UTF-8
 ctype    en_US.UTF-8
 tz       America/Los_Angeles
 date     2025-01-30
 pandoc   3.2 @ /Applications/RStudio.app/Contents/Resources/app/quarto/bin/tools/aarch64/ (via rmarkdown)

- Packages -------------------------------------------------------------------
 package      * version date (UTC) lib source
 abind          1.4-8   2024-09-12 [1] CRAN (R 4.4.1)
 bit            4.0.5   2022-11-15 [1] CRAN (R 4.4.0)
 bit64          4.0.5   2020-08-30 [1] CRAN (R 4.4.0)
 car          * 3.1-3   2024-09-27 [1] CRAN (R 4.4.1)
 carData      * 3.0-5   2022-01-06 [1] CRAN (R 4.4.0)
 cellranger     1.1.0   2016-07-27 [1] CRAN (R 4.4.0)
 cli            3.6.3   2024-06-21 [1] CRAN (R 4.4.0)
 colorspace     2.1-1   2024-07-26 [1] CRAN (R 4.4.0)
 crayon         1.5.3   2024-06-20 [1] CRAN (R 4.4.0)
 digest         0.6.37  2024-08-19 [1] CRAN (R 4.4.1)
 dplyr        * 1.1.4   2023-11-17 [1] CRAN (R 4.4.0)
 evaluate       1.0.3   2025-01-10 [1] CRAN (R 4.4.1)
 fansi          1.0.6   2023-12-08 [1] CRAN (R 4.4.0)
 fastmap        1.2.0   2024-05-15 [1] CRAN (R 4.4.0)
 fitdistrplus * 1.2-2   2025-01-07 [1] CRAN (R 4.4.1)
 forcats      * 1.0.0   2023-01-29 [1] CRAN (R 4.4.0)
 Formula        1.2-5   2023-02-24 [1] CRAN (R 4.4.0)
 generics       0.1.3   2022-07-05 [1] CRAN (R 4.4.0)
 glue         * 1.8.0   2024-09-30 [1] CRAN (R 4.4.1)
 here         * 1.0.1   2020-12-13 [1] CRAN (R 4.4.0)
 hms            1.1.3   2023-03-21 [1] CRAN (R 4.4.0)
 htmltools      0.5.8.1 2024-04-04 [1] CRAN (R 4.4.0)
 jsonlite       1.8.8   2023-12-04 [1] CRAN (R 4.4.0)
 kableExtra   * 1.4.0   2024-01-24 [1] CRAN (R 4.4.0)
 knitr          1.48    2024-07-07 [1] CRAN (R 4.4.0)
 lattice        0.22-6  2024-03-20 [1] CRAN (R 4.4.0)
 lifecycle      1.0.4   2023-11-07 [1] CRAN (R 4.4.0)
 lmtest         0.9-40  2022-03-21 [1] CRAN (R 4.4.0)
 magrittr       2.0.3   2022-03-30 [1] CRAN (R 4.4.0)
 MASS         * 7.3-64  2025-01-04 [1] CRAN (R 4.4.1)
 Matrix         1.7-0   2024-03-22 [1] CRAN (R 4.4.0)
 munsell        0.5.1   2024-04-01 [1] CRAN (R 4.4.0)
 pillar         1.9.0   2023-03-22 [1] CRAN (R 4.4.0)
 pkgconfig      2.0.3   2019-09-22 [1] CRAN (R 4.4.0)
 pscl         * 1.5.9   2024-01-31 [1] CRAN (R 4.4.1)
 R6             2.5.1   2021-08-19 [1] CRAN (R 4.4.0)
 readr        * 2.1.5   2024-01-10 [1] CRAN (R 4.4.0)
 readxl       * 1.4.3   2023-07-06 [1] CRAN (R 4.4.0)
 rlang          1.1.4   2024-06-04 [1] CRAN (R 4.4.0)
 rmarkdown      2.28    2024-08-17 [1] CRAN (R 4.4.0)
 rprojroot      2.0.4   2023-11-05 [1] CRAN (R 4.4.0)
 rstudioapi     0.16.0  2024-03-24 [1] CRAN (R 4.4.0)
 scales         1.3.0   2023-11-28 [1] CRAN (R 4.4.0)
 sessioninfo  * 1.2.2   2021-12-06 [1] CRAN (R 4.4.0)
 stringi        1.8.4   2024-05-06 [1] CRAN (R 4.4.0)
 stringr        1.5.1   2023-11-14 [1] CRAN (R 4.4.0)
 survival     * 3.7-0   2024-06-05 [1] CRAN (R 4.4.0)
 svglite        2.1.3   2023-12-08 [1] CRAN (R 4.4.0)
 systemfonts    1.1.0   2024-05-15 [1] CRAN (R 4.4.0)
 tibble         3.2.1   2023-03-20 [1] CRAN (R 4.4.0)
 tidyselect     1.2.1   2024-03-11 [1] CRAN (R 4.4.0)
 tzdb           0.4.0   2023-05-12 [1] CRAN (R 4.4.0)
 utf8           1.2.4   2023-10-22 [1] CRAN (R 4.4.0)
 vcd          * 1.4-13  2024-09-16 [1] CRAN (R 4.4.1)
 vctrs          0.6.5   2023-12-01 [1] CRAN (R 4.4.0)
 viridisLite    0.4.2   2023-05-02 [1] CRAN (R 4.4.0)
 vroom          1.6.5   2023-12-05 [1] CRAN (R 4.4.0)
 withr          3.0.2   2024-10-28 [1] CRAN (R 4.4.1)
 xfun           0.48    2024-10-03 [1] CRAN (R 4.4.1)
 xml2           1.3.6   2023-12-04 [1] CRAN (R 4.4.0)
 yaml           2.3.10  2024-07-26 [1] CRAN (R 4.4.0)
 zoo            1.8-12  2023-04-13 [1] CRAN (R 4.4.0)

 [1] /Library/Frameworks/R.framework/Versions/4.4-arm64/Resources/library

------------------------------------------------------------------------------
\end{verbatim}




\end{document}
